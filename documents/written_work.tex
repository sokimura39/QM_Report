% Options for packages loaded elsewhere
\PassOptionsToPackage{unicode}{hyperref}
\PassOptionsToPackage{hyphens}{url}
\PassOptionsToPackage{dvipsnames,svgnames,x11names}{xcolor}
%
\documentclass[
  letterpaper,
  DIV=11,
  numbers=noendperiod]{scrartcl}

\usepackage{amsmath,amssymb}
\usepackage{iftex}
\ifPDFTeX
  \usepackage[T1]{fontenc}
  \usepackage[utf8]{inputenc}
  \usepackage{textcomp} % provide euro and other symbols
\else % if luatex or xetex
  \usepackage{unicode-math}
  \defaultfontfeatures{Scale=MatchLowercase}
  \defaultfontfeatures[\rmfamily]{Ligatures=TeX,Scale=1}
\fi
\usepackage{lmodern}
\ifPDFTeX\else  
    % xetex/luatex font selection
\fi
% Use upquote if available, for straight quotes in verbatim environments
\IfFileExists{upquote.sty}{\usepackage{upquote}}{}
\IfFileExists{microtype.sty}{% use microtype if available
  \usepackage[]{microtype}
  \UseMicrotypeSet[protrusion]{basicmath} % disable protrusion for tt fonts
}{}
\makeatletter
\@ifundefined{KOMAClassName}{% if non-KOMA class
  \IfFileExists{parskip.sty}{%
    \usepackage{parskip}
  }{% else
    \setlength{\parindent}{0pt}
    \setlength{\parskip}{6pt plus 2pt minus 1pt}}
}{% if KOMA class
  \KOMAoptions{parskip=half}}
\makeatother
\usepackage{xcolor}
\setlength{\emergencystretch}{3em} % prevent overfull lines
\setcounter{secnumdepth}{-\maxdimen} % remove section numbering
% Make \paragraph and \subparagraph free-standing
\ifx\paragraph\undefined\else
  \let\oldparagraph\paragraph
  \renewcommand{\paragraph}[1]{\oldparagraph{#1}\mbox{}}
\fi
\ifx\subparagraph\undefined\else
  \let\oldsubparagraph\subparagraph
  \renewcommand{\subparagraph}[1]{\oldsubparagraph{#1}\mbox{}}
\fi


\providecommand{\tightlist}{%
  \setlength{\itemsep}{0pt}\setlength{\parskip}{0pt}}\usepackage{longtable,booktabs,array}
\usepackage{calc} % for calculating minipage widths
% Correct order of tables after \paragraph or \subparagraph
\usepackage{etoolbox}
\makeatletter
\patchcmd\longtable{\par}{\if@noskipsec\mbox{}\fi\par}{}{}
\makeatother
% Allow footnotes in longtable head/foot
\IfFileExists{footnotehyper.sty}{\usepackage{footnotehyper}}{\usepackage{footnote}}
\makesavenoteenv{longtable}
\usepackage{graphicx}
\makeatletter
\def\maxwidth{\ifdim\Gin@nat@width>\linewidth\linewidth\else\Gin@nat@width\fi}
\def\maxheight{\ifdim\Gin@nat@height>\textheight\textheight\else\Gin@nat@height\fi}
\makeatother
% Scale images if necessary, so that they will not overflow the page
% margins by default, and it is still possible to overwrite the defaults
% using explicit options in \includegraphics[width, height, ...]{}
\setkeys{Gin}{width=\maxwidth,height=\maxheight,keepaspectratio}
% Set default figure placement to htbp
\makeatletter
\def\fps@figure{htbp}
\makeatother

\KOMAoption{captions}{tableheading}
\makeatletter
\makeatother
\makeatletter
\makeatother
\makeatletter
\@ifpackageloaded{caption}{}{\usepackage{caption}}
\AtBeginDocument{%
\ifdefined\contentsname
  \renewcommand*\contentsname{Table of contents}
\else
  \newcommand\contentsname{Table of contents}
\fi
\ifdefined\listfigurename
  \renewcommand*\listfigurename{List of Figures}
\else
  \newcommand\listfigurename{List of Figures}
\fi
\ifdefined\listtablename
  \renewcommand*\listtablename{List of Tables}
\else
  \newcommand\listtablename{List of Tables}
\fi
\ifdefined\figurename
  \renewcommand*\figurename{Figure}
\else
  \newcommand\figurename{Figure}
\fi
\ifdefined\tablename
  \renewcommand*\tablename{Table}
\else
  \newcommand\tablename{Table}
\fi
}
\@ifpackageloaded{float}{}{\usepackage{float}}
\floatstyle{ruled}
\@ifundefined{c@chapter}{\newfloat{codelisting}{h}{lop}}{\newfloat{codelisting}{h}{lop}[chapter]}
\floatname{codelisting}{Listing}
\newcommand*\listoflistings{\listof{codelisting}{List of Listings}}
\makeatother
\makeatletter
\@ifpackageloaded{caption}{}{\usepackage{caption}}
\@ifpackageloaded{subcaption}{}{\usepackage{subcaption}}
\makeatother
\makeatletter
\@ifpackageloaded{tcolorbox}{}{\usepackage[skins,breakable]{tcolorbox}}
\makeatother
\makeatletter
\@ifundefined{shadecolor}{\definecolor{shadecolor}{rgb}{.97, .97, .97}}
\makeatother
\makeatletter
\makeatother
\makeatletter
\makeatother
\ifLuaTeX
  \usepackage{selnolig}  % disable illegal ligatures
\fi
\IfFileExists{bookmark.sty}{\usepackage{bookmark}}{\usepackage{hyperref}}
\IfFileExists{xurl.sty}{\usepackage{xurl}}{} % add URL line breaks if available
\urlstyle{same} % disable monospaced font for URLs
\hypersetup{
  pdftitle={Correlation between Cycle Hire Trips and the Terrain},
  colorlinks=true,
  linkcolor={blue},
  filecolor={Maroon},
  citecolor={Blue},
  urlcolor={Blue},
  pdfcreator={LaTeX via pandoc}}

\title{Correlation between Cycle Hire Trips and the Terrain}
\author{}
\date{}

\begin{document}
\maketitle
\ifdefined\Shaded\renewenvironment{Shaded}{\begin{tcolorbox}[borderline west={3pt}{0pt}{shadecolor}, interior hidden, enhanced, sharp corners, boxrule=0pt, frame hidden, breakable]}{\end{tcolorbox}}\fi

\hypertarget{qm-written-assignment}{%
\section{QM Written assignment}\label{qm-written-assignment}}

\hypertarget{introduction}{%
\subsection{Introduction}\label{introduction}}

The London Cycle Hire Scheme (LCHS), known as Santander Cycles, is a
dock-based bike share scheme that operates in London since 2010
(\textbf{li2019?}), now having 800 docking stations within central
London (\textbf{transportforlondon2023a?}). Users can use the bikes for
journeys between docking stations, allowing for casual usage of bicycles
for short, high-frequency, and one-way journeys (\textbf{beecham2015?};
\textbf{beroud2012?}). The usage data is publicly available, enabling
analysis of cycling behaviour.

\hypertarget{literature-review}{%
\subsection{Literature Review}\label{literature-review}}

The detailed datasets of cycle hire schemes around the world has enabled
analysis for cycling behaviour from many aspects.
(\textbf{gebhart2014?}) has explored how weather conditions affect the
number of bikeshare trips in Washington DC, finding uncomfortable
conditions including cold weather, precipitation, and high humidity
reduces bikeshare usage.

Research on the relationship between cycling behaviour and the physical
environment is limited. (\textbf{rodriguez2004?}) explored the impacts
of the physical environment such as the topography and the existence of
sidewalks and bike paths to modal choices, and have found that a
positive slope (uphill) reduces the odds of walking or cycling. This
research surveyed the students and staff at a university, which may show
a different behaviour from the general public.

Drawing on these two streams, this report has conducted an anlysis on
the less-explored relationship between cycle hire scheme usage behaviour
and the physical environment.

\hypertarget{research-question}{%
\subsection{Research Question}\label{research-question}}

Is there a correlation between the quantity or duration of cycle hire
trips and the slope of the physical environment?

\hypertarget{hypothesis}{%
\subsection{Hypothesis}\label{hypothesis}}

The trips involving uphill paths will be discouraged, thus leading to a
smaller number of trips done by the cycle hire scheme.For the trips that
do occur, the duration will be significantly longer compared to their
downhill counterparts. Journeys made by ebikes are less disencouraged by
uphills compared to classic bikes.

\hypertarget{data}{%
\subsection{Data}\label{data}}

\hypertarget{data-sources}{%
\subsubsection{Data Sources}\label{data-sources}}

Transport for London (TfL) makes data for the LCHS publicly available.
The location of the 800 docking stations is available from the TfL
Unified API (\textbf{transportforlondon2023a?}), where real time
availability is updated along with the name, ID, and the number of docks
each station has. Usage data is shared on their website as well
(\textbf{transportforlondon2023?}), with the ID and name for start and
end stations, date and time, duration, and type of cycle (conventional
or e-cycle) for every journey taken by the LCHS.

The spatial data is joined by the physical attributes from other
resources to extract the required information. LIDAR Composite Digital
Terrain Model (DTM) (\textbf{environmentalagency2023?}) with a
resolution of 2 m was used for extracting elevation. Statistical
boundaries and congestion charge boundaries used are from the London
Datastore (\textbf{greaterlondonauthority2014?};
\textbf{greaterlondonauthority2019?}).

\hypertarget{summary-of-data}{%
\subsubsection{Summary of data}\label{summary-of-data}}

The scope of analysis is the journeys taken on the LCHS from December
2022 to November 2023, the most recent one-year period where data is
available. There is a total of 8,414,631 journeys taken during this
period, of which .

The summary of the data is shown as follows.

Statistics \textbar{} Classic Cycles \textbar{} E-cycles \textbar{}
Total \textbar{}

\textbar===\textbar---\textbar---\textbar---\textbar{} \textbar{} Number
of Trips \textbar{} 7,808,234 \textbar{} 606,397 \textbar{} 8,414,631
\textbar{} \textbar{} Mean Travelled Distance {[}m{]} \textbar{} 2279.41
\textbar{} 3124.62 \textbar{} 2340.32 \textbar{} \textbar{} Mean Height
Difference {[}m{]} \textbar{} -0.23 \textbar{} -0.07 \textbar{} -0.22
\textbar{}

As a whole, the average height difference between the origin and
destination is -0.22 metres, a slight negative value indicating downhill
travel is slightly preferred over their uphill counterparts. Different
behaviour can be observed for the classic cycles and e-cycles in terms
of distance and height difference. Longer distance and the smaller
difference in height infer e-cycles allows journeys more physically
demanding to the cyclists. From this comparison between the classic
cycles and e-cycles, we believe that the physical intensity of cycling
may have a correlation on the journeys taken on the LCHS.

There are many factors that influence the choice of travel mode, and we
cannot immediately conclude the height difference is the decisive
factor. This report will attempt to identify the effect of the height
difference on the journeys. From this point, our focus will be on
journeys made by classic cycles, which

\hypertarget{spatial-distribution-of-cycling-behaviour}{%
\subsubsection{Spatial Distribution of Cycling
Behaviour}\label{spatial-distribution-of-cycling-behaviour}}

The spatial characteristics (although not the main focus of this report)
is an important aspect of cycling behaviour. We will define Central
London as the congestion charge zone (\textbf{transportforlondon?})
which is roughly the area within the Ring Road. The following table
shows the characteristics of journeys, classified by their relation to
the Central London area.

\textbar{} Outside \textbar\textbar{} Inside \textbar\textbar{}\\
\textbar{} Outside \textbar{} Inside \textbar{} Outside \textbar{}
Inside \textbar{}

\textbar===\textbar---\textbar---\textbar---\textbar---\textbar{}
\textbar{} Number of Trips \textbar{} 3,788,352 \textbar{} 1,432,552
\textbar{} 1,265,977 \textbar{} 1,927,750 \textbar{} \textbar{} Mean
Travelled Distance {[}m{]} \textbar{} 1972.20 \textbar{} 3290.51
\textbar{} 3294.99 \textbar{} 1730.68 \textbar{} \textbar{} Mean Height
Difference {[}m{]} \textbar{} -0.17 \textbar{} -0.04 \textbar{} -0.73
\textbar{} -0.11 \textbar{}

\hypertarget{methodology}{%
\subsection{Methodology}\label{methodology}}

\hypertarget{joining-and-cleaning-data}{%
\subsubsection{Joining and Cleaning
Data}\label{joining-and-cleaning-data}}

Using these datasets, we have conducted a spatial join to get the
elevation and the LSOA of each docking station, and classified whether
the station is within the congestion charge area. This docking station
dataset with the additional information is joined with the usage data,
allowing us to calculate the distance, the difference in height, and the
origin-destination combination for LSOAs. The average height difference
for e-bikes were -0.07 m, significantly (\(p < 0.01\)) lower than that
of e-bikes.

\hypertarget{the-impacts-of-height-difference}{%
\subsubsection{The impacts of height
difference}\label{the-impacts-of-height-difference}}

First, we will consider whether there is a correlation between the
height difference travelled and the number of journeys. The descriptive
statistics of these will be as follows:

The negative number in the difference in height indicates that downhill
travel is more encouraged as a general trend. The one-sample t-test
against the null hypothesis of average being 0 shows a p-value of
\(p < 0.001\), showing this is statistically significant. Both
conventional cycles and e-bikes have negative average height difference,
with e-bikes having less average showing

\hypertarget{regression-model}{%
\subsubsection{Regression Model}\label{regression-model}}

In order to analyse the impact of the height difference, a multiple
regression model is conducted to measure the impacts of this variable.

\hypertarget{results}{%
\subsection{Results}\label{results}}

The results for the simple regression model is as follows:

\hypertarget{discussion}{%
\subsection{Discussion}\label{discussion}}

\hypertarget{conclusion}{%
\subsection{Conclusion}\label{conclusion}}



\end{document}
